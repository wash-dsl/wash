\chapter{Introduction}

Smooth Particle Hydrodynamics (SPH) is a mesh-free Lagrangian method used to obtain approximate solutions to fluid dynamics problems by simulating a large number of particles and their interactions \cite{liu2010smoothed}. This has been used extensively in scientific and industrial applications. These applications include simulations of dam breaking \cite{monaghan1994simulating} and rigid bodies such as a ship hull impacting water \cite{veen2010smoothed}. SPH can also be easily extended to other problems such as gases or dust fluids \cite{monaghan1995sph}, and thermal and matter diffusion. It is extensively used in the special effects and graphics industry, including video games, movies and TV advertising \cite{monaghan2005sph}.

Due to the practical nature of the SPH problem, many software frameworks and libraries have been developed. In order to obtain meaningful results from SPH simulations, a very large number of particles must be considered. Additionally, the equations describing the particles' behaviour are fairly complex. Hence, large-scale simulations require massive amounts of computing power, far beyond the capabilities of a typical desktop PC. Because of this, high-performance computing nodes or clusters are used to perform such simulations. Such systems are characterised by high levels of parallelism, which requires the simulation code to be written in a way that can fully utilise these parallel systems.

Writing programs for high-performance systems is complex, as it requires in-depth knowledge of the target system in order to fully exploit its performance. This results in high-performance programs being expensive to develop and difficult to maintain. To address this issue, many libraries have been developed that aim to facilitate this process. Such libraries are often specific to a specialised problem domain and target platform. One of the techniques to making the code more readable and easier to write is using a Domain-Specific Language (DSL). 

A DSL is a specialised programming language that includes convenient syntax for describing common operations in a specific problem domain. Since SPH simulations typically have similar structure and use a common class of operations, they are well-suited to be described using a specially-designed DSL. This is why we decided to use this technique to develop a framework that aids in creating SPH programs. This approach has another benefit of enabling the user to write platform-independent code, since the framework handles parallelism and applies problem and architecture specific optimisation techniques.
