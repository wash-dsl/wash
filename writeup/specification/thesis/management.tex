\chapter{Project Management} % Methodology, Architecture, Risk management

\section{Methodology}

In order to manage the project, the team will utilise a Scrumban methodology \cite{ladas2009scrumban}. This agile methodology combines the system of sprint cycles typical of a scrum approach with the use of a kanban board to keep track of tasks \cite{scrum2023scrum, kanban2023agile}.  

The project will be split into week-long sprints. At the beginning of each sprint, a scrum meeting will take place (attended by all team members and the project supervisor) in order to specify what tasks need to be completed during the week, and who these tasks will be assigned to. The project supervisor's input will be used to guide this process. If necessary, the project team will meet separately afterwards in order to debrief and decide exactly what tasks need to be completed, and whose responsibility each task will be.

Project supervisor meetings will also serve as sprint retrospectives for the previous sprint. To this end, the successes and failures of the previous week will be discussed in these meetings. Any blocking tasks will also be identified (the supervisor may be able to give advice as to how progress can be made towards these), and the overall methodology will also be evaluated and modified as appropriate.

A kanban board will be used to record new tasks as they arise, and to keep track of the execution of individual objectives. Columns of the kanban board will be used to track the progression of each objective through the execution cycle, with special markers being used to indicate any tasks that have become blocked by another task. Swimlanes will be used to rank each task according to its priority. All team members will have access to the kanban board, and will ensure that it is consulted and updated frequently, in order to ensure that it accurately describes the current state of the project's execution.


\section{Roles}

Each member of the team is expected to be deeply involved in all aspects of the project, including developing software, communicating our results, and planning sprint cycles. However, the team felt it was appropriate to define the following roles, in order to delegate some more specific responsibilities:

\begin{itemize}
    \item \textbf{Project Manager (\textit{Tom White}):} Tom's role will be to direct the execution of the project. He will be ultimately responsible for delegating tasks between the other team members, and making any key decisions (particularly if there is significant disagreement amongst the rest of the team about a specific decision).

    \item \textbf{Academic Specialist (\textit{Scott Parker}):} Scott's role will be to interface with academics and their research in order to ensure that the team has access to up-to-date knowledge of developments in the field. He will also take responsibility for planning meetings with the project supervisor (and other experts in this area) as appropriate.

    \item \textbf{Customer Specialist (\textit{Maciej Waszczuk}):} Maciej's role will be to interface with the customer in order to ensure that the outputs of the project satisfy their expectations. He will also take responsibility for planning meetings with the customer (and other external stakeholders) as appropriate.

    \item \textbf{DevOps Specialist (\textit{Miko\l{}aj W\k{a}sacz}):} Miko\l{}aj's role will be to ensure that development best-practices are adhered to by the rest of the team, particularly concerning version control. He will also be responsible for setting up environments (writing makefiles, installing libraries, etc.). 

    \item \textbf{Testing and Validation Specialist (\textit{James Macer-Wright}):} James will be primarily responsible for designing the testing and validation systems that will be used throughout development. Ensuring that the simulations generated by our DSL are correct is critical, and having a specific team member aiming to ensure this specifically was deemed appropriate. He will also be responsible for evaluating and comparing the performance of our DSL simulations to ensure that they are within targets.

    \item \textbf{Documentation Specialist (\textit{Tom Divers}):} Tom will be primarily responsible for the documentation and other artefacts produced to communicate the progression and results of the project. Specifically, this includes the specification,\footnote{This just so happens to be the document you are reading currently.} the progress presentation, the final report and the final presentation. % TODO: not sure if this footnote is necessary
\end{itemize}

The team has also identified the following external stakeholders that are expected to have a significant impact on the direction of the project:
\begin{itemize}
    \item \textbf{Project Supervisor (\textit{Dr Gihan Mudalige}):} Dr Mudalige is the principle academic responsible for advising and assessing this project. He has a wealth of experience in developing active-library DSLs for HPC applications 
    % TODO: Is Gihan a Dr or Prof
    % TODO: As this is a formal document, I think we should write Dr Mudalige instead of Gihan
    (specifically OPS \cite{reguly2014ops} and OP2 \cite{mudalige2012op2}), and first introduced the team to SPH simulations. He is also responsible for teaching Warwick's CS325 Compiler Design module (which all team members took in 2022), further evidencing his knowledge and experience in this area.  

    \item \textbf{Customer (\textit{Dr Richard Kirk}):} Dr Kirk is designated as the official customer of this project. In practice, he will also serve as a source of expertise for the team members. He also has experience in HPC (evidenced by him teaching CS402 High-Performance Computing), and can also be trusted for advice. 
\end{itemize}

\section{Timeline}

In order to plan the work effectively, we have created a provisional timetable, which is shown as a Gantt chart in figure \ref{fig:ganttchart1}. This timetable will be used as a reference to track progress and to ensure that the project is finished on time. However, if complications arise during development, the schedule may be adjusted slightly.

\begin{figure}[h]
\begin{center}
\hspace*{-1.8cm}    
\begin{ganttchart}[y unit title=0.4cm,
y unit chart=0.5cm,
vgrid,hgrid, 
title label anchor/.style={below=-1.6ex},
title left shift=.05,
title right shift=-.05,
title height=1,
progress label text={},
bar height=0.7,
group right shift=0,
group top shift=.6,
group height=.3]{1}{28}
%labels
\gantttitle{Month}{28} \\
\gantttitle{October}{4} 
\gantttitle{November}{4} 
\gantttitle{December}{4} 
\gantttitle{January}{4} 
\gantttitle{February}{4} 
\gantttitle{March}{4}
\gantttitle{April}{4} \\
%tasks
\ganttbar[progress=0]{\footnotesize Specification}{1}{1}\\
\ganttbar[progress=0]{\footnotesize Design}{2}{4} \\
\ganttbar[progress=0]{\footnotesize Serial Implementation}{5}{10} \\
\ganttbar[progress=0]{\footnotesize OpenMP Implementation}{11}{14} \\
\ganttbar[progress=0]{\footnotesize CUDA Implementation}{15}{18} \\
\ganttbar[progress=0]{\footnotesize Additional Backend, if time allows}{19}{21}\\
\ganttbar[progress=0]{\footnotesize Testing \& Validation}{22}{23}\\
\ganttbar[progress=0]{\footnotesize Final Report}{7}{28}\\
\end{ganttchart}
\end{center}
\caption{Project Gantt Chart.}
\label{fig:ganttchart1}
\end{figure}



